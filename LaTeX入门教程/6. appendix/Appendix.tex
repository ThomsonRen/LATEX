\documentclass[12pt]{elegantpaper} % 源文件的类型

%------------------------------------------------------
%             宏包区
%------------------------------------------------------
\usepackage{appendix}
\usepackage{listings}
% 代码段的格式设置
% language 使用的语言
% basicstyle 字号
% numbers 行号的的位置,可取值none, left, right
% keywordstyle 关键词的颜色大小
% numberstyle 行号的颜色大小
% stepnumber 行号从几开始
% numbersep 设置行号与代码之间的间隔
% commentstyle 注释的颜色大小
% backgroundcolor 背景的颜色
% frame 外框的格式, shadow / single / none
% framexleftmargin 外框在左边的页边距
% tabsize 将默认tab设置为x个空格
% rulecolor 边框颜色
% title 设置代码块的标题是文件的标题
% escapeinside 用于中文注释乱码

\lstset{language=[ANSI]c,
	basicstyle=\small,
	numbers=left,
	keywordstyle=\color{blue},
	numberstyle={\tiny\color{lightgray}},
	stepnumber=1,
	numbersep=5pt,
	commentstyle=\small\color{red},
	backgroundcolor=\color[rgb]{0.95,1.0,1.0},
	frame=shadowbox,
	framexleftmargin=5mm,
	tabsize=4,
	rulecolor=\color{black},
	title=\lstname,
	escapeinside=``}

\usepackage{xcolor}

%\usepackage[vlined,boxed,linesnumbered,commentsnumbered]{algorithm2e}
\usepackage[ruled,vlined]{algorithm2e}

% boxed: to have algorithms enclosed in a box.
% ruled: to have algorithms with a line at the top and the bottom
% commentsnumbered: makes comments be numbered if numbering is active.
% lined: 代码中for-end等组合之间的连线
% vlined:代码中for-end等组合之间的连线末尾有一短横线


%% 文章信息区
\title{\LaTeX:附录的插入与使用}
\author{Yushu @ LinCol}
\date{\today}

%% 文档开始
\begin{document}
	
\maketitle
	
\newpage
	
\section{Introduction}

%\appendix
%\appendixpage

\newpage
\begin{appendices}
		
\section{First appendix}

Here are simulation programmes we used in our model as follow.\\

\textbf{\textcolor[rgb]{0.98,0.00,0.00}{Input matlab source:}}

\lstinputlisting[language=Matlab]{./code/mcmthesis-matlab1.m}

\section{Second appendix}

some more text \textcolor[rgb]{0.98,0.00,0.00}{\textbf{Input C++ source:}}
\lstinputlisting[language=C++]{./code/mcmthesis-sudoku.cpp}
\lstinputlisting[language=Python,firstline=1,lastline=15]{./code/PSO.py}

\begin{lstlisting}[language=C]
int main(int argc, char ** argv) { 
printf("Hello world!\n"); // `显示结果`
return 0; 
}
\end{lstlisting}

\section{伪代码的插入}

\begin{algorithm}
	\SetKwData{Left}{left}\SetKwData{This}{this}\SetKwData{Up}{up}
	\SetKwFunction{Union}{Union}\SetKwFunction{FindCompress}{FindCompress}
	\KwIn{A bitmap $Im$ of size $w\times l$}
	\KwOut{A partition of the bitmap}
	\BlankLine
	\emph{special treatment of the first line}\;
	\For{$i\leftarrow 2$ \KwTo $l$}{
		\emph{special treatment of the first element of line $i$}\;
		\For{$j\leftarrow 2$ \KwTo $w$}{\label{forins}
			\Left$\leftarrow$ \FindCompress{$Im[i,j-1]$}\;
			\Up$\leftarrow$ \FindCompress{$Im[i-1,]$}\;
			\This$\leftarrow$ \FindCompress{$Im[i,j]$}\;
			\If(\tcp*[h]{O(\Left,\This)==1}){\Left compatible with \This}{\label{lt}
				\lIf{\Left $<$ \This}{\Union{\Left,\This}}
				\lElse{\Union{\This,\Left}}
			} % \tcp*[h] 居左注释 \tcp*[f] 居右注释
			\If(\tcp*[f]{O(\Up,\This)==1}){\Up compatible with \This}{\label{ut}
				\lIf{\Up $<$ \This}{\Union{\Up,\This}}
				\tcp{\This is put under \Up to keep tree as flat as possible}\label{cmt}
				\lElse{\Union{\This,\Up}}
			}
		}
		\lForEach{element $e$ of the line $i$}{\FindCompress{p}}
	}
	\caption{disjoint decomposition}\label{algo_disjdecomp}
\end{algorithm}

\begin{algorithm}
	\KwData{$G=(X,U)$ such that $G^{tc}$ is an order.}
	\KwResult{$G’=(X,V)$ with $V\subseteq U$ such that $G’^{tc}$ is an
		interval order.}
	\Begin{
		$V \longleftarrow U$\;
		$S \longleftarrow \emptyset$\;
		\For{$x\in X$}{
			$NbSuccInS(x) \longleftarrow 0$\;
			$NbPredInMin(x) \longleftarrow 0$\;
			$NbPredNotInMin(x) \longleftarrow |ImPred(x)|$\;
		}
		\For{$x \in X$}{
			\If{$NbPredInMin(x) = 0$ {\bf and} $NbPredNotInMin(x) = 0$}{
				$AppendToMin(x)$}
		}
		\nl\While{$S \neq \emptyset$}{\label{InRes1}
			\nlset{REM} remove $x$ from the list of $T$ of maximal index\;\label{InResR}
			\lnl{InRes2}\While{$|S \cap ImSucc(x)| \neq |S|$}{
				\For{$ y \in S-ImSucc(x)$}{
					\{ remove from $V$ all the arcs $zy$ : \}\;
					\For{$z \in ImPred(y) \cap Min$}{
						remove the arc $zy$ from $V$\;
						$NbSuccInS(z) \longleftarrow NbSuccInS(z) - 1$\;
						move $z$ in $T$ to the list preceding its present list\;
						\{i.e. If $z \in T[k]$, move $z$ from $T[k]$ to
						$T[k-1]$\}\;
					}
					$NbPredInMin(y) \longleftarrow 0$\;
					$NbPredNotInMin(y) \longleftarrow 0$\;
					$S \longleftarrow S - \{y\}$\;
					$AppendToMin(y)$\;
				}
			}
			$RemoveFromMin(x)$\;
		}
	}
	\caption{IntervalRestriction\label{IR}}
\end{algorithm}

% \nl 编号
% \nlset 自己设置编号格式
% \lnl
引用伪代码中的内容:\ref{InResR}, \ref{InRes2}
\end{appendices}

\end{document}

