% 参考文献的引用
% 一 将文献附在文章之后
% 二 .bib文件结构

\documentclass{article}

%% 文章信息区
\title{\LaTeX: edit in tex files}
%\author{任公子@SJTU}
\date{\today}


\begin{document}

\maketitle

\newpage

This  system can be used for many types of airplanes\cite{1}, and it also solves the interference during  the procedure of the boarding airplane,as described above we can get to the  optimization boarding time\cite{sze:liu}.We also know that all the service is automate.\cite[page 22]{2}

% \begin{thebibliography}{width} 中的width用以限制参考文献序号的宽度

% 99 意味着不超过两位数字。通常设定为与参考文献的数目一致。
\begin{thebibliography}{99}
\bibitem{1} D.~E. KNUTH   The \TeX{}book  the American
Mathematical Society and Addison-Wesley
Publishing Company , 1984-1986.
\bibitem{2}Lamport, Leslie,  \LaTeX{}: `` A Document Preparation System '',
Addison-Wesley Publishing Company, 1986.
\bibitem{sze:liu}
Szegedy C, Liu W, Jia Y, Sermanet P, Reed S, Anguelov D, Erhan D, Vanhoucke V, Rab-inovich A.:
Going deeper with convolutions. In:
Proceedings of the 2015 IEEE Conference on Computer Vision and Pattern Recognition,
Cvpr 2015. IEEE, Boston (2015).
\end{thebibliography}

\end{document}
