% latex 数学符号与公式


% 本教程2019-7-20 前由 Yushu Xu 维护
% 2019-7-20 起由 Tongxin Ren  维护

\documentclass[12pt]{elegantpaper} % 源文件的类型

%============================================================
%                     宏包区
%============================================================

% 数学类
\usepackage{amsmath}        %数学公式

% 颜色
\usepackage{xcolor}
\numberwithin{equation}{section}
%============================================================
%                     标题区
%============================================================
\title{\LaTeX : 数学符号与公式}

% 作者
\author{任公子@SJTU}

% 日期
\date{\today}  

%============================================================
%                    文档开始
%============================================================
\begin{document}

\maketitle   %生成标题页

%============================================================
\section{常用数学符号}


上下标: $A_{ij}^{2}$ %用^ 和_ 标明上下标。注意上下标的内容(子公式)一般需要用花括号包裹,否则上下标只对后面的一个符号起作用。


求和: $\sum_{i=1}^{N} x_i$

完整求和符号可在行间公式展开: $$\sum_{i=1}^{N} x_i$$

或者使用displaystyle命令  $\displaystyle \sum_{k=1}^{N} a_k^2$



比较符号:$\textless$, $\le$, $\leq$, $\textgreater$, $\ge$, $\geq$, $\geqq$, $\approx$, $\neq$


矩阵:

$
\begin{pmatrix}
0 & 23 & 1 \\

1 & 0 \\

3 & 7
\end{pmatrix}
$



%  以下部分不再多介绍

无穷大: $\infty$

加减号:$\pm$, $\mp$

属于:$\in$

积分: $\int_0^x f(x) dx$, $\oint$

乘号: $\times$, $\cdot$, $\prod$

加减号:$\pm$, $\mp$

分式:$\dfrac{a_i}{b_i+c_i}$

根号:$\sqrt[6]{a}$

二项式结构:

$\binom{a}{b}$


西文省略号

1, 2, $\ldots$, 10

1, 2, $\dots$, 10

1+2+3+$\cdots$+10

$\vdots$

$\ddots$


\section{希腊字母}

$\alpha, \beta, \gamma, \theta, \phi, \lambda, \delta, \triangle, \sigma$

$\Gamma, \Theta, \Phi, \Lambda, \Delta, \Sigma$

\section{不能直接输入的标点符号}

1200 \%

1 数学模式符号:
\$

$a+b$

2 注释符号:
\%

3 上标:
\^{}

4 下标:
\_

5 取反符号:
\~{}

6 宏命令:
\textbackslash

7 宏定义符号:
\#

8 表格对齐符号:
\&

9 分组:
\{ \}



%============================================================
\section{数学公式的插入与编辑}


\subsection{行内公式}
%与文字混排
This equation $x + y = z$ is always right.





\subsection{行间公式}
%单独占一行

$$x+y=z$$


% 或者写为
\[x+y=z\]


无编号:
\begin{equation*}
x+y=z
\end{equation*}


有编号:
\begin{equation}
\label{eq1}
x+y=z
\end{equation}



两种不同的引用方式:
As we can see from the above equation \ref{eq1}, 

%\usepackage{amsmath}
As we can see from the above equation \eqref{eq1}, 



自定义公式的编号:
\begin{equation}
\tag{Eq.1}
x+y=z
\end{equation}

\subsection{多行公式}
一个编号
\begin{multline}
a+b  \\
c+d \\
e+f 
\end{multline}


\begin{equation}
\begin{aligned}
a  & = b  + c \\
& = d + e  \\
& = f + g
\end{aligned}
\end{equation}

对齐,多编号
\begin{align}
\label{eq2}
a & = b + c \\
& = d + e
\end{align}

无对齐,多编号
\begin{gather}
a+b=2 \\
c+d+e=2 \\
e+f=1
\end{gather}
%============================================================
\section{【强推】公式识别神奇:mathpix}



\end{document}
