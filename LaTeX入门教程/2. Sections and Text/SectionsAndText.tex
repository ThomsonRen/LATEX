% latex的文本与段落结构

% 本教程2019-7-20 前由 Yushu Xu 维护
% 2019-7-20 起由 Tongxin Ren  维护

\documentclass[12pt, a4paper, onecolumn]{elegantpaper}  % 源文件的类型

% [] 可选参数,{}必选参数


% 文件类型: article,report,book,slides

% 字号的设置:10pt / 11pt / 12pt

% 长度的单位
	% pt:point
	% in: inch 英寸 (1in = 72.27pt)
	% cm (2.54cm = 1in)
	% em (字母M的长度)

% 纸张大小:a4paper / letterpaper

% 排版格式:onecolumn / twocolumn


%============================================================
%                     宏包区
%============================================================

% 设置全局字体
\usepackage{txfonts}        %全文定义罗马字体Time New Roman
%\usepackage{helvet}
%\usepackage{courier}        %打字机(等宽)字体

\usepackage{amsthm}          %特殊文本环境

% 颜色
\usepackage{color}          %文本颜色
\usepackage{xcolor}

\usepackage{geometry}       %页面格式
\geometry{ 
	left=2cm,
	right=2cm,
	top=2cm, 
	bottom=2cm,
}






%============================================================
%                     标题区
%============================================================
\title{\LaTeX : 文本与段落结构}
% 作者
\author{任公子@SJTU}
\date{}

%日期
\date{\today}  % 自动显示现在的日期
%\date{2019.01.22}
%\date{Jan. $20^{th}$}   % 需使用amsmath宏包





%============================================================
%                    文档开始
%============================================================
\begin{document}
 
\maketitle  %生成标题页,包含标题、摘要和日期等信息

% 修改摘要的名字
%\renewcommand{\abstractname}{摘要} % abstract -> Summary
\begin{abstract}
这是我的摘要。这是我的摘要。这是我的摘要。这是我的摘要。这是我的摘要。这是我的摘要。这是我的摘要。这是我的摘要。

\end{abstract}



%另起新页 
\newpage 
%\clearpage

\tableofcontents % 目录
\newpage
%============================================================

\textcolor{red}{\Huge{章节的划分:}}

\section{SECTION}
iMoDS model consists of three main submodels, Risk Cost Model, Power Supply Ability Model, and Series Safety Model. For explanation, there is a figure which connects all related parts together.

\subsection{subsection}
iMoDS model consists of three main submodels, Risk Cost Model, Power Supply Ability Model, and Series Safety Model. For explanation, there is a figure which connects all related parts together.

\subsubsection{subsubsection}
iMoDS model consists of three main submodels, Risk Cost Model, Power Supply Ability Model, and Series Safety Model. For explanation, there is a figure which connects all related parts together.


%============================================================
\textcolor{red}{\Huge{空格,分行与分段方法}}

(1) 空格

this is a brief\quad document


(2) 分行

this is a brief document about writing \\ this is a 

this is a brief document about writing \newline this is a 

(3) 分段

首行缩进

this is a brief document about writing. I'd like to introduce it to you.

this is a brief document about writing. I'd like to introduce it to you.

首行不缩进

\noindent this is a brief document about writing. I'd like to introduce it to you.

%============================================================
\textcolor{red}{\Huge{更改字体、字号、颜色}}

字体

\textit{Introduction}  % 意大利体

\textup{Introduction}  % 直立体

\textsl{Introduction}  % 斜体

\textsc{Introduction}  % 全部大写

\textbf{Introduction}  % 加粗体

颜色

\textcolor{red}{Introduction}

\textcolor{blue}{Introduction}

\textcolor{green}{Introduction}

突出强调

You \emph{should} use it carefully.

You \textbf{should} use it carefully.

You \underline{should} use it carefully.

字体大小(由小到大)

\tiny apple

\scriptsize apple

\footnotesize apple

\small apple

\normalsize apple

\large apple

\Large apple

\LARGE apple

\huge apple

\normalsize
 
%============================================================
\textcolor{red}{\Huge{文本对齐方式}}

\begin{center}
iMoDS model consists of three main submodels, Risk Cost Model, Power Supply Ability Model, and Series Safety Model. For explanation, there is a figure which connects all related parts together.
\end{center}

\begin{flushleft}
iMoDS model consists of three main submodels, Risk Cost Model, Power Supply Ability Model, and Series Safety Model. For explanation, there is a figure which connects all related parts together.
\end{flushleft}

\begin{flushright}
iMoDS model consists of three main submodels, Risk Cost Model, Power Supply Ability Model, and Series Safety Model. For explanation, there is a figure which connects all related parts together.
\end{flushright}

%============================================================

\end{document}
